\documentclass{article}
\usepackage{amsmath}
\usepackage{amssymb}
\usepackage{enumerate}
\title{Math Quiz - Brent Questions}

\begin{document}
\maketitle 

\begin{itemize}
  \item[2.] The answer to each of the following is zero, one, or
    infinity.  Choose wisely.

    \begin{enumerate}[(a)]
      \item What does the following expression evaluate to?
        \[
          \lim_{n \to \infty} \frac{n!e^{n}}{2\pi n^{n + \frac{1}{2}}}
          \int_{-\infty}^{\infty} \left[
            \sum_{k=0}^{\infty} 
            \frac{1}{k!}
            \left(\frac{-x^2}{2}\right)^k
          \right]
          \,\mathrm{d}x
        \]

      \item Consider the system of linear equations.
        \[
          \begin{array}{rcrcrcrcr}
            w   &+& 3x  &+& 2y  &+& 2z  &=& 0 \\
            w   &+& 4x  &+& y   &+&     &=& 0 \\
            3w  &+& 5x  &+& 10y &+& 14z &=& 0 \\
            2w  &+& 5x  &+& 5y  &+& 6z  &=& 0
          \end{array}
        \]
        How many solutions $(w, x, y, z)$ are there (where $w, x, y, z$
        are real)?

      \item Let $y$ is a real-valued function defined on the real line
        and satisfying the initial value problem below:
        \begin{align*}
          y' + xy &= x \\
          y(0) &= -1
        \end{align*}
        Find
        \[
          \lim_{x \to -\infty} y(x)
        \]

      \item Let $G = (V, E)$ be the graph with $V = \mathbb{R}$, and
        $E = \{\{x, y\} : x, y \in \mathbb{R} \}$, i.e., $G$ is the
        complete graph on the reals.  Suppose each edge in $E$ is
        colored either red or blue.  Call a subgraph $H \subseteq G$
        ``monochromatic'' if all of its edges are the same color.  How
        many monochromatic, complete subgraphs $H \subseteq G$ must one
        find such that $|V(H)| = |\mathbb{R}|$?

      \item The function $f : \mathbb{R} \to \mathbb{R}$ is defined as
        follows:
        \[
          f(x) = \begin{cases}
            3x^2 &\text{if } x \in \mathbb{Q} \\
            -5x^2 &\text{if } x \notin \mathbb{Q}
          \end{cases}
        \]
        At how many places is $f$ differentiable?
    \end{enumerate}

    Solutions
    \begin{enumerate}[(a)]
      \item 1 (it's the integral of the normal distribution, cleverly
        disguised).
      \item $\infty$ (Math GRE practice 24)
      \item 1 (Math GRE practice 44)
      \item 0.  Ramsey theory; trust me.
      \item 1 (Math GRE practice 47)
    \end{enumerate}

\pagebreak

  \item[5.] ``Paradoxes''.  Match the following with their common names:
    \begin{enumerate}[(a)]
      \item The expression ``the smallest positive integer definable in
        under sixty letters''
      \item An adjective is autological if and only if it describes
        itself, e.g., `English', 'unhyphenated', 'pentasyllabic'.
        
        An adjective is heterological if and only if it does not
        describe itself, e.g., `hyphenated' and 'monosyllabic'.

        Into which category does `heterological' fall?  
      \item Let $R = \{x : x \notin x\}$.  Then $R \in R \iff R \notin
        R$.
      \item ``All Cretans are liars.'' --- a Cretan
      \item If this sentence is true, then Germany borders China.
      \item The graph of $x \mapsto 1/x$, rotated about the $x$-axis
        gives a finite volume, but the outer shell has infinite surface
        area.
      \item When the Okies left Oklahoma and moved to California, they
        reaised the average intelligence level in both states.
      \item ``yields falsehood when preceded by its quotation.'' yields
        falsehood when preceded by its quotation.
      \item Consider the following infinite set of sentences:
        \begin{itemize}
          \item[$(S_1)$] For each $i > 1$, $S_i$ is not true.
          \item[$(S_2)$] For each $i > 2$, $S_i$ is not true.
          \item[$(S_3)$] For each $i > 3$, $S_i$ is not true.
          \item[$\vdots$]
        \end{itemize}
      \item ``Some numbers are squares, while others are not; therefore,
        all the numbers, including both squares and non-squares must be
        more numerous than just the squares.  And yet, for every square
        there is exactly one positive number that is its square root,
        and for every number there is exactly one square; hence there
        cannot be more of one than of the other.''
    \end{enumerate}

    Solution set:

    \begin{enumerate}[(1)]
      \item Painter's paradox
      \item The Epimenides paradox
      \item Berry paradox
      \item Russell's paradox
      \item Quine's paradox
      \item Curry's paradox
      \item Grelling--Nelson paradox
      \item Will Rogers phenomenon
      \item Galileo's paradox
      \item Yablo's paradox
    \end{enumerate}


    Answers:
    \begin{enumerate}[(a)]
      \item 3
      \item 7
      \item 4
      \item 6
      \item 2
      \item 1
      \item 8
      \item 5
      \item 10
      \item 9
    \end{enumerate}


\end{itemize}
\end{document}
