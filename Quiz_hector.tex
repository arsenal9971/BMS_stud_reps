\documentclass[12pt,spanish]{article}
\usepackage[spanish]{babel}
\usepackage{amsmath}
\title{Math Quiz - Hector Questions}

\begin{document}
\maketitle 

\begin{enumerate}
\item[1.] The next words represent numbers in different languages, relate each word to its correspondent number and write down the sum of them (some of the words represent the same number, and some numbers have no correspondent word).

\textbf{- Words:}
\begin{enumerate}
\item[i)] once
\item[ii)] ichi
\item[iii)] to
\item[iv)] vingt
\item[v)] en
\item[vi)] se
\item[vii)] dvadeset
\item[viii)] saan
\item[ix)] dreizig
\end{enumerate}
\textbf{- Numbers:}
\begin{enumerate}
\item[a)] 3
\item[b)] 11
\item[c)] 20
\item[d)] 15
\item[e)] 21
\item[f)] 30
\item[g)] 2
\item[h)] 1
\item[i)] 16
\end{enumerate}
- Answers: once - 11 (spanish),  ichi - 1 (japanese), to - 2 (norwegian), vingt - 20 (french), en - 1 (danish), se - 3 (corean), dvadeset - 20 (croata) saan - 3 (chino), dreizig - 30 (german).

- Answers translated to items: i)-b),ii)-h),iii)-g),iv)-c),v)-h),vi)-a),vii)-c),viii)-a),ix)-f).

\item[4.] Which of the next problems are P, NP-Complete and Not know (NP-intermedate). 
\begin{enumerate}
\item[i)] Travelling Salesman.
\item[ii)] Factoring integers.
\item[iii)] Group isomorphism problem.
\item[iv)] Hamiltonian path.
\item[v)] Maximum matching.
\item[vi)] Eulerian path.
\item[vii)] Graph isomorphism.
\end{enumerate}
- Answers: i) NP-complete, ii) NP-intermedate, iii) NP-intermedate, iv) NP-complete, v)- P, vi)- P, vii)- P, viii)- NP-compete.

\item[5.] Mathematicians and Physicist who had no university studies in mathematics or physics. 
\begin{enumerate}
\item[i)] Edward Witten.
\item[ii)] Leonhard Euler.
\item[iii)] Joseph Louis Lagrange.
\item[iv)] George Green.
\item[v)] Blaise Pascal.
\item[vi)] Isaac Newton.
\item[vii)] Oliver Heaviside.
\item[viii)] Srinivasa Ramanujan.
\item[ix)] Gottfried Wilhelm Leibniz.
\end{enumerate} 
- Answers: i) No university studies in physics, he was journalist. ii) Studies in Mathematics. iii) Studies in  Matheamatics, iv) No university studies in math, self-taught mathematician, v) No university studies in math, vi) Studies in Mathematics, vii) No studies in math, self-taught, viii) No studies in math, self-taught, viii) No studies in math, ix) No studies in math.

\textbf{- Final stage:}
Question for the winning team: 

\textbf{A Question of Dancing ( from book Algebra Can be Fun, Perelman):} At a party, 20 people danced. Mary danced with seven parthners, Olga with eight, Vera with nine, and so forth up to Nina who danced with all the partners. How many men partners were there at the party?.

\text{Solution:} This is a very simple problem in the unknown is suitable chosen. Let us seek the number of girls rather than men: the number of girls is x:
\begin{itemize}
\item 1st, Mary danced with 6+1 partners,
\item 2nd, Olga danced with 6+2 partners,
\item 3rd, Vera danced with 6+3 partners,
\item ...
\item xth, Nina danced with 6+x partners.
\end{itemize}
We get the following equation,
$$
x + (6+x) = 20
$$
from which we find that $x = 7$ and hence that there were $20-7 = 13$ men at the party.
\end{enumerate}
\end{document}
